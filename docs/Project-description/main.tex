%----------------------------------------------------------------------------------------
%	PACKAGES AND OTHER DOCUMENT CONFIGURATIONS
%----------------------------------------------------------------------------------------

\documentclass[11pt]{scrartcl} % Font size

\input{structure.tex} % Include the file specifying the document structure and custom commands
\usepackage{multirow}
\usepackage{array}
\usepackage{subcaption}
\usepackage{textcomp}
\usepackage{algorithm}
\usepackage{algpseudocode}

\usetikzlibrary{positioning}

% Define a macro to create a table with fixed column widths
\newcolumntype{C}[1]{>{\centering\arraybackslash}p{#1}}

\usepackage{hyperref}
\hypersetup{
    colorlinks=true,
    linkcolor=blue,
    filecolor=magenta,      
    urlcolor=cyan,
}

\definecolor{diffstart}{named}{Apricot}
\definecolor{diffincl}{named}{Green}
\definecolor{diffrem}{named}{Red}

\usepackage{listings}
  \lstdefinelanguage{diff}{
    basicstyle=\ttfamily\small,
    morecomment=[f][\color{diffstart}]{@@},
    morecomment=[f][\color{diffincl}]{+\ },
    morecomment=[f][\color{diffrem}]{-\ },
  }

\definecolor{codegreen}{rgb}{0,0.6,0}
\definecolor{codegray}{rgb}{0.5,0.5,0.5}
\definecolor{codepurple}{rgb}{0.58,0,0.82}
\definecolor{backcolour}{rgb}{0.95,0.95,0.92}
\definecolor{codeblue}{rgb}{0,0,0.8}

\lstdefinestyle{mystyle}{
    backgroundcolor=\color{backcolour},   
    commentstyle=\color{codegreen},
    keywordstyle=\color{codeblue},
    numberstyle=\tiny\color{codegray},
    stringstyle=\color{codepurple},
    basicstyle=\ttfamily\footnotesize,
    breakatwhitespace=false,         
    breaklines=true,                 
    captionpos=b,                    
    keepspaces=true,                 
    numbers=left,                    
    numbersep=5pt,                  
    showspaces=false,                
    showstringspaces=false,
    showtabs=false,                  
    tabsize=2
}
\usepackage{tocloft}
\renewcommand{\cftsecfont}{\normalfont}
\renewcommand{\cftsecpagefont}{\normalfont}
\addto\captionsgreek{\renewcommand{\contentsname}{\normalfont Περιεχόμενα}}

\lstset{style=mystyle}

%----------------------------------------------------------------------------------------
%	TITLE SECTION
%----------------------------------------------------------------------------------------

\title{	
	\normalfont\normalsize
	\textsc{Πανεπιστήμιο Πατρών, Τμήμα Μηχανικών ΗΥ και Πληροφορικής \\Τεχνολογία λογισμικού 2023}\\ % Your university, school and/or department name(s)
	\vspace{25pt} % Whitespace
	\rule{\linewidth}{0.5pt}\\ % Thin top horizontal rule
	\vspace{20pt} % Whitespace
    {\Large Project Description v0.1}\\ % The assignment title
	\vspace{12pt} % Whitespace
	\rule{\linewidth}{0.5pt}\\ % Thick bottom horizontal rule
	\vspace{12pt} % Whitespace
    \includegraphics[width=0.7\textwidth]{../../brand/png/logo-transparent.png}
        \rule{\linewidth}{2pt}
}
\author{
Κωνσταντίνος Γιακαλλής \\UP1072533 \and Ιωάννης Παναρίτης \\UP1072632 \and \hspace{4ex} Βασίλειος Τσούλος \\ \hspace{5ex}UP1072605 \and \hspace{1ex} Νικόλαος Χαλκιόπουλος \\ \hspace{1ex} UP1072572
}



\date{} % Today's date (\today) or a custom date

%----------------------------------------------------------------------------------------
%	DOCUMENT
%----------------------------------------------------------------------------------------

\bibliographystyle{ieeetr}
\addto\captionsgreek{\renewcommand{\refname}{Αναφορές}}


\begin{document}

\maketitle
\pagebreak
\Large

Τίτλος: \src{Maintena}

\section*{Project Description}

\subsection*{Περιγραφή}

Το \src{Maintena} είναι μια φιλική προς το χρήστη εφαρμογή για κινητά που στοχεύει να διευκολύνει τη συντήρηση του οχήματος για τους οδηγούς. Κάθε όχημα έχει ένα εγχειρίδιο σέρβις από τον κατασκευαστή του που αναφέρει ποιο, πόσο συχνά και πώς πρέπει να συντηρείται το όχημα. Το Maintena παρέχει μια εύχρηστη λύση που βοηθά τους οδηγούς να διατηρήσουν τη μακροζωία του οχήματός τους και την ομαλή λειτουργία του.

\subsection*{Χαρακτηριστικά}
\begin{itemize}
    \item Υπενθυμίσεις συντήρησης - Η εφαρμογή θα ελέγχει συνεχώς τα χιλιόμετρα που διανύθηκαν (ή τις ώρες χρήσης για ορισμένες συσκευές) και θα παρέχει στον χρήστη τακτικές ενημερώσεις σχετικά με το πότε θα γίνει η επόμενη υπηρεσία, ποιες αλλαγές πρέπει να γίνουν, πόσο πιθανόν θα κοστίσει και πού να προμηθευτείτε ανταλλακτικά ή τη βοήθεια έμπειρου τεχνίτη.
    \item Αναβαθμίσεις απόδοσης - Το Maintena προσφέρει προτάσεις για αναβαθμίσεις για τη βελτίωση της απόδοσης του οχήματος, όπως αλλαγές φίλτρων, χαρτογράφηση εγκεφάλου και αγωνιστικά καύσιμα. Αυτές οι προτάσεις βασίζονται στις οδηγικές συνήθειες του χρήστη και στην κατάσταση του οχήματος.
    \item Ιστορικό προβλημάτων - Η εφαρμογή παρακολουθεί όλα τα ζητήματα που αντιμετώπισε το όχημα στο παρελθόν, συμπεριλαμβανομένης της ημερομηνίας του προβλήματος, του κόστους επισκευής και της τοποθεσίας του συνεργείου επισκευής.
    \item Νέος πίνακας συντήρησης - Η εφαρμογή προσαρμόζει το πρόγραμμα συντήρησης στα νέα εξαρτήματα που είναι εγκατεστημένα στο όχημα. Για παράδειγμα, ένας κινητήρας που λειτουργεί με μεγαλύτερη συμπίεση από την κανονική απαιτεί συχνότερη και καλύτερη συντήρηση.
    \item Community Discussion - Maintena παρέχει μια πλατφόρμα για τους χρήστες να συζητούν και να λύνουν σχετικές ερωτήσεις με άλλους οδηγούς και τεχνίτες.
    \item Διάγνωση οχήματος - Η εφαρμογή προσφέρει καλύτερη επίγνωση της κατάστασης και των δυνατοτήτων του οχήματος. Παρέχει μια διαγνωστική αναφορά που εντοπίζει πιθανά προβλήματα και προτείνει λύσεις για τη διατήρηση της λειτουργίας του οχήματος με τον καλύτερο δυνατό τρόπο.
\end{itemize}

\subsection*{Οφέλη}
    \begin{itemize}
        \item Άνεση - Το Maintena παρέχει έναν βολικό τρόπο παρακολούθησης του προγράμματος συντήρησης του οχήματός σας και παρέχει προτάσεις για βελτιώσεις.
        \item Οικονομία - Η εφαρμογή βοηθά στη διατήρηση της μακροζωίας του οχήματος, μειώνοντας την ανάγκη για δαπανηρές επισκευές και αντικαταστάσεις.
        \item Βελτιωμένη απόδοση - Η εφαρμογή προσφέρει προτάσεις για αναβαθμίσεις που μπορούν να βελτιώσουν την απόδοση του οχήματος, κάνοντας την οδήγηση πιο απολαυστική.
        \item Κοινότητα - Η εφαρμογή παρέχει μια πλατφόρμα στους χρήστες να συζητούν και να λύνουν σχετικές ερωτήσεις με άλλους οδηγούς και τεχνίτες.
    \end{itemize}

% \bibliography{bibliography}

\end{document}
